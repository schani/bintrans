\documentclass{article}
\usepackage{a4}
\usepackage{endnotes}

\newcommand{\enref}[1]{(\ref{#1})}
\newcommand{\enoteref}[1]{$^{\ref{#1}}$}
\newcommand{\ir}[1]{\texttt{#1}}
\newcommand{\cc}[1]{\texttt{#1}} %c code

\title{Specification}
%i suppose we need a snappy name for this library.  any ideas?
\author{Mark Probst}
\date{}

\begin{document}

\maketitle

\section{Overview}

The unit of compilation is called ``fragment''.  A fragment has one
entry point and one or more exit points.  It consists of one or more
basic blocks.  Only forward control flow is allowed\endnote{This
restriction may be lifted in future versions.  The reason for it is
that it allows us to do second-chance binpacking without doing a DFA
for the \texttt{ARE\_CONSISTENT} vector.  We can simply
(conservatively) assume that a spill store for a virtual register is
necessary if it has been written during its livetime (i.e. while being
assigned to a register).} within fragments.  Basic blocks are basic
blocks, i.e., they have one entry point and one exit point, which may
be an unconditional or a two-way branch.

The IR is flat and RISC-like.  The variables used in the instruction
stream are called ``virtual registers'', or short ``virtual''.  There
are two types of virtual registers:

\begin{itemize}
\item Guest registers.  They correspond to the registers of the guest.

\item Temp registers.  They are created by the front-end to hold
    temporary values.
\end{itemize}

From the point of view of the front-end the difference between guest
and temp registers is that the set of guest registers is
fixed\endnote{Guest registers in binary translation correspond to
variables in high-level language compilation.  We should not forget
that people might want to use different variable sets in different
fragments.}, whereas temp registers can be created at arbitrary times
during fragment creation.

Each virtual register has a type.  The available standard types are:

\begin{itemize}
\item Integer.

\item Float.

\item Condition.  Conditions are one-bit virtual variables.  Only they
    can be used in conditional branches.
\end{itemize}

Integers usually have the biggest size that the host allows without
compromising speed.  This means that integers have the host's word
size.  Floats are 64 bits IEEE format\endnote{Floating point
operations are provided for single and double precision.}.

Additional types can be defined by the user.  Such virtual registers
will be correctly allocated by the register allocator, but can only be
used in user-defined operations\endnote{Such virtual registers can be
used to implement instruction set extensions like SSE2.}.

There are several types of instructions:

\begin{itemize}
\item Constant load instructions.  These copy constant values into
    virtual registers.

\item Copy instructions.  They copy a values between registers of the
    same type.

\item Loads/Stores.  A load takes an address (as a virtual register) and
    assigns the value at that address to a virtual register.  A store
    takes an address (as a virtual) and a value and stores the value
    at that address in memory.  A store has no left-hand side, i.e.,
    has no result.  There are loads and stores for each type and each
    width (1,2,4,8 for integers, 4,8,10 for floats).  Conditions
    cannot be stored directly, but must be converted to integers
    first.  Widths exceeding the sizes of the virtual registers are
    not supported, i.e., there is no 8 byte integer store instruction
    on the i386.  Integer loads whose size is less than the virtual
    register width leave the upper part of the virtual register
    undefined\endnote{Different architectures have different
    conventions regarding this.  The Alpha, for example, sign extends
    4 byte results, but zero extends 2 byte and 1 byte results.  The
    x86-64 zero extends all results.  Using one or the other means to
    take a performance hit on architectures which don't do it our way.
    It should not matter anyway, since for all operations for which it
    matters there are variants for all different widths.  See also
    note~\enref{en:widths}.\label{en:undefined}}.  Float loads with
    smaller width result in values of the correct width.

\item Conversions.  They take a register of one type and return a
    register of another type.  There are conversions for
    integer$\rightarrow$float, float$\rightarrow$integer and
    condition$\rightarrow$integer.

\item Predicates.  A predicate takes one or more registers of the same
    type and returns a condition, indicating whether the registers
    satisfy the predicate.  There are variants for each integer
    operand width.

\item ALU/FP operations.  They take one or more registers of the same
    type and assign a value to a register of that type.  For some
    operations\endnote{I see no reason why we should have, for
    example, bit operations for each operand width.  The decision not
    to have them is of course that there is no canonical
    representation for values with a width less than the virtual
    register width.  I don't see this as a
    disadvantage.\label{en:widths}} there are variants for each
    operand width.  Integer operations with a width less than the
    virtual register width leave the upper part of the result
    undefined\enoteref{en:undefined}.  FP operations
    result in values of the correct widths.

\item Unconditional branch.  Branches to a basic block in the same
    fragment.

\item Conditional branch.  Takes a condition and branches to a specified
    basic block in the same fragment if the condition is true.  If it
    is false, falls through to the next basic block in the fragment.

\item Exits.  Leaves the fragment.  An exit makes the fragment behave
    like an ordinary procedure taking no arguments by using the
    call/return convention of the host, but disregarding the
    caller-saved/callee-saved convention\endnote{We will some day
    implement full calling convention compliance, but for now we
    assume all registers which we use are caller-saved.  That means
    that the user code must either use glue code to call fragments or
    user-defined operations must do the register loading and saving at
    entries and exits of fragments.}.

\item Procedure calls.  These use the standard calling convention of the
    host.  Guest registers are automatically saved to their home
    locations, live temp registers are spilled.  Arguments to the
    function are integer virtuals and their number is fixed\endnote{Is
    it necessary to have a more general argument handling system?  If
    so, what would be a sensible way to do this?}.  The address of the
    procedure to be called is passed in a virtual
    register\endnote{This will usually be a bit host-dependent.
    Usually you will want to use an address stored in a table pointed
    to by some fixed register.  It will be simple matter though, to
    let a host-independent front-end call-back some short
    host-dependent function to generate that address.}.

\item User-defined operations.  These are inherently host-dependent and
    must be implemented by the user\endnote{I am very sure that I will
    need this, mostly for calling assembler routines with non-standard
    calling conventions.  I want to keep the interface as simple as
    possible as well as pose little overhead for the code generator.}.
    A user-defined operations can take arguments (virtual registers)
    and can produce a result.  The front-end can specify whether all
    guest registers must be saved to their home locations before the
    user-defined operation.  A user-defined operation can be an exit.
\end{itemize}

\section{IR Instructions}

\subsection{Constant Loads}

Constant loads are available for all widths of integers and floats.
Integer loads leave the upper part of the left-hand side virtual
undefined.  Constant loads for conditionals are available in the form
of the operations clear, which sets a condition to zero, and set,
which sets it to one.

\begin{verbatim}
T1 :=4 123
C2 := 0
\end{verbatim}

\subsection{Copies}

There are copy operations for integers and floats.  They copy the
value of one virtual to another.  Copies do not distinguish between
widths.  There are no copies for conditionals\endnote{Would we need
them?}.

\begin{verbatim}
T1 := T2
\end{verbatim}

\subsection{Loads and Stores}

Loads and stores are available for all widths of integers and floats.
Integer loads leave the upper part of the virtual assigned to in an
undefined state.  The addresses of the memory locations are given as
virtuals.  There are no loads and stores for conditions.

Loads and stores are provided for both byte orders, i.e., little
endian and big endian.

\begin{verbatim}
T1 :=4 MEM_LE[T2]
MEM_BE[T3] :=4 T4
\end{verbatim}

\subsection{Conversions}

Conversion operations are provided for converting between integers and
floats and from conditions to integers.  To avoid combinatorial
explosion of operators, conversions always operate on the whole
virtual register width.  In order to convert from a smaller integer
width, the value in question must be sign or zero extended.  The
result of a conversion to an integer is undefined if the virtual
register width is not large enough to hold the result.  Conversions
from or to floats always operate on signed integers.  Converting a
condition to an integer sets the integer virtual to either 1 or 0,
depending on whether the condition is true or false.

\begin{verbatim}
T1 := INT(F1)
F2 := FLOAT(T3)
T4 := INT(C5)
\end{verbatim}

\subsection{Predicates}

Comparison predicates compare two values of the same type (integer or
float) and produce a condition value corresponding on the outcome of
the comparison.  In addition, there are predicates for discovering
properties of integers.

Comparison predicates for floats always operate on the whole virtual
register values, i.e., with the highest precision.  Integer predicates
are available for all widths, if the behaviour differs between widths.

The following comparisons are available for floats: equal, less than.

The comparisons for integers are: equal, less than unsigned, less than
signed, add carry, subtract carry, overflow.

The predicates on single integers available for all widths are: zero,
parity even, sign\endnote{Maybe we should remove the sign and even
operations and instead just provide a ``lowest bit set'' operation.
Sign and even can then be easily synthesized via extract, negate and
LBS.\label{en:lbs}}

The predicates on single integers operating on the whole virtual
register width are: even\enoteref{en:lbs}.

Other predicates can be synthesized by swapping operands and by
negating the resulting condition.

\begin{verbatim}
C1 := LESSU4(T1,T2)
C3 := EQUAL(F4,F5)
C6 := ZERO(T7)
\end{verbatim}

\subsection{Integer ALU Operations}

Most integer ALU operations do not distinguish between operand widths
but operate on the whole virtual register width.  Only those
operations which give different results for different widths, i.e.,
those where more signifant bits of inputs have an effect on less
significant bits in the result, are provided for all widths.

The following ALU operations operate on all bits: Add, negate,
subtract, bitwise and, bitwise or, bitwise xor, bitwise negate.

These ALU operations have variants for all widths: Shift
left\endnote{The operand width in the case of shift left is only
important and necessary for the shift amount.}, logic shift right,
arithmetic shift right, sign extend, zero extend.

%The ALU operation swap bytes is available for all width other than 1.
%%NOTE: I don't think we need this, since we have loads and stores for
%%both byte orders.

The two operations extract and insert are special, inasmuch as they
are the only operations that take arguments which are not virtual
registers.  Extract extracts a bit-field from a word, given its
starting bit and its length.  Both these numbers are immediate
arguments to the operation\endnote{I expect extract to be a frequent
operation, especially when the guest is the i386.  A sequence of more
primitive operations semantically equal to extract is 4 operations
long (two shifts and two constant loads to load the shift amounts).}.
Insert does the opposite of extract.  Given two virtual registers and
the two immediates describing a bit-field (start and length), it
inserts into the first virtual register at the given bit-field
position the contents of the second virtual register, which must be
right-aligned.

Sign and zero extend extend from the specified width to the whole
virtual register width.  

There are no rotate operations.  Rotations must be implemented via
shifts and bitwise ors.

\begin{verbatim}
T1 := ADD(T2,T3)
T4 :=4 LSHIFTR(T5,T6)
T7 := ZEX2(T8)
T9 := EXTRACT(T10,0,8)
T11 := INSERT(T12,T13,8,8)
\end{verbatim}

\subsubsection{Multiply, Divide, and Modulo}

Most architectures provide instructions for doubleword multiplication
and division, so we have to support this, too.  The following
operations are provided for all widths\endnote{It probably makes sense
to provide this only for widths 4 and 8.}:

\begin{description}
\item[\ir{r :=w MUL(a,b)}] Multiply \ir{a} with \ir{b} and
assign the lower half of the result to \ir{r}.

\item[\ir{r :=w MULHS(a,b)}, \ir{r :=w MULHU(a,b)}] Sign/zero extend
\ir{a} and \ir{b} to twice the width of the operation, multiply them,
and assign the upper half of the result to \ir{r}.

\item[\ir{r :=w DIVS(ah,al,b)}, \ir{r :=w DIVU(ah,al,b)},
\ir{r :=w MODS(ah,al,b)}, \ir{r :=w MODU(ah,al,b)}] Form a value twice
the width of the operation by using \ir{ah} as the upper and \ir{al} as
the lower half.  Calculate the quotient/modulo of that value and the
sign/zero extension of \ir{b} to the same width, and assign the lower
half of the result to \ir{r}.
\end{description}

\subsection{Condition ALU Operations}

The following operations are provided for conditions: negate, and, or,
xor.

\begin{verbatim}
C1 := NEG(C2)
C3 := AND(C4,C5)
\end{verbatim}

\subsection{FP Operations}

Some FP operations distinguish between single and double precision.

The following FP operations distinguish between single and double
precision: add, subtract, multiply, divide, square root\endnote{That's
all that most architectures need.  However, what do we do with with
i387's stupid additional operations, like cosine?  I think it would be
best to handle that via user-ops, like the MMX stuff.}.

These operations work on both single and double precision numbers:
negate, absolute value.

\begin{verbatim}
F1 :=d ADD(F2,F3)
F4 := ABS(F5)
\end{verbatim}

\subsection{Unconditional Branch}

An unconditional branch transfers control to a basic block in the same
fragment.

\begin{verbatim}
GOTO BB4
\end{verbatim}

\subsection{Conditional Branches}

A conditional branch transfers control to one of two basic blocks in
the same fragment, based on the value of a condition.

The current implementation requires that the basic block targeted by
the condition-false edge directly follows the basic block with the
branch in sequential order.

\begin{verbatim}
IF C1 THEN GOTO BB2 ELSE GOTO BB3
\end{verbatim}

\subsection{User-Defined Operations}

\subsubsection{Exits}

\section{API}

The API only provides for generation, inspection and compilation of IR
code.  It does not provide means for modifying IR code.  Applications
doing optimizations on a fragment have to create a new fragment
containing the optimized IR code\endnote{As far as I could see, this
is the way Valgrind does it.  I think it's a good start.  We might
provide IR manipulation functions in the future, though.}.

\subsection{Memory Management}

The data structures for fragments are allocated from arenas.  An arena
is a preallocated memory block from which smaller blocks can be
allocated in a linear fashion.  Memory allocated from an arena is not
freed until the whole arena is freed.  This makes memory management
both fast and very simple.

\begin{verbatim}
arena_t* make_arena (size_t size);
\end{verbatim}

Allocates and returns an arena with \cc{size} bytes using \cc{malloc}.

\begin{verbatim}
void dispose_arena (arena_t *arena);
\end{verbatim}

Frees the memory occupied by the arena \cc{arena}.

\begin{verbatim}
void* arena_alloc (arena_t *arena, size_t size, size_t alignment);
\end{verbatim}

Allocates and returns a block of \cc{size} bytes aligned to a multiple
of \cc{alignment} bytes from \cc{arena}.  If the required amount of
memory cannot be allocated (because the remaining free space in the
arena is too small), \cc{0} is returned.

\subsection{Generation}

\begin{verbatim}
fragment_t* make_fragment (arena_t *arena);
\end{verbatim}

Create a new fragment in \cc{arena} which is initially empty.

\begin{verbatim}
bblock_t* make_bblock (fragment_t *fragment);
\end{verbatim}

Create a new basic block to be part of \cc{fragment}.

\begin{verbatim}
void emit_bblock (fragment_t *fragment, bblock_t *bblock);
\end{verbatim}

Append the basic block to the fragment.  Code can only be emitted to a
basic block after the block has been emitted and before any other
block has been emitted to the fragment\endnote{This should make memory
management for IR instructions much easier and reduce memory
requirements for them.  Objections?}.

\begin{verbatim}
vreg_t* make_treg (fragment_t *fragment, reg_type_t type);
\end{verbatim}

Introduces a new temp register in \cc{fragment}.  Its type is given by
\cc{type}.  The built-in types are \cc{REG\_TYPE\_INTEGER},
\cc{REG\_TYPE\_FLOAT}, and \cc{REG\_TYPE\_CONDITION}.  Additional
types can be defined by the user (see section~\ref{sec:regdef}).  The
new register is is assumed to be dead from the start\endnote{This
means that if liveness information is to be provided via
\cc{set\_vreg\_live}, the register must be set live explicitly before
its first definition.  The reason for making a register dead by
default is that if it were the other way around, and a register would
not be live from the start on, there would nevertheless be an
unnecessary short live range at the start, unless we were to use
clever tricks.}.

\begin{verbatim}
void set_vreg_live (fragment_t *fragment, vreg_t *reg, int is_live);
\end{verbatim}

Set the liveness status of the virtual register \cc{reg} in
\cc{fragment} to live if \cc{is\_live} is non-zero, to dead
otherwise\endnote{The use of this function will not be mandatory.
Instead, an additional liveness analysis pass will be provided.  That
pass will of course be unnecessary if liveness is annotated via this
function.}.  A virtual register must be set to live before emitting
the instruction producing a value in that register and must be set
dead after emitting the instruction which uses the value last.
Setting a virtual register dead and live again without emitting
intervening instructions introduces a new live range\endnote{In our
implementation this means that the two live ranges can be allocated to
different host registers even though they are consecutive.}.  A
register which has been set to live at least once must be set to dead
before or at the end of the fragment and not be set to live again.

\begin{verbatim}
void emit_ir_int1_const (bblock_t *bblock, vreg_t *lhs,
                         ir_int1_t value);
void emit_ir_int2_const (bblock_t *bblock, vreg_t *lhs,
                         ir_int2_t value);
void emit_ir_int4_const (bblock_t *bblock, vreg_t *lhs,
                         ir_int4_t value);
void emit_ir_int8_const (bblock_t *bblock, vreg_t *lhs,
                         ir_int8_t value);
void emit_ir_float4_const (bblock_t *bblock, vreg_t *lhs,
                           ir_float4_t value);
void emit_ir_float8_const (bblock_t *bblock, vreg_t *lhs,
                           ir_float8_t value);
\end{verbatim}

Append a load constant instruction to the end of \cc{bblock}.  These
may be implemented as macros.

\begin{verbatim}
void emit_ir_int_load4 (bblock_t *bblock, vreg_t *lhs, vreg_t *addr);
void emit_ir_int_lessu4 (bblock_t *bblock, vreg_t *lhs,
                         vreg_t *arg1, vreg_t *arg2);
...
\end{verbatim}

Append a simple instruction (all operands are virtual registers) to
the end of \cc{bblock}.  The two functions above are just given as
examples, since all functions are not listed in this document.  All
the simple emits will be implemented as macros using these four
functions\endnote{There are variants for each number of arguments so
that we don't do unnecessary copying of dead values.}:

\begin{verbatim}
void emit_ir_op0 (bblock_t *bblock, ir_op_t op, vreg_t *lhs);
void emit_ir_op1 (bblock_t *bblock, ir_op_t op, vreg_t *lhs,
                  vreg_t *arg1);
void emit_ir_op2 (bblock_t *bblock, ir_op_t op, vreg_t *lhs,
                  vreg_t *arg1, vreg_t *arg2);
void emit_ir_op3 (bblock_t *bblock, it_op_t op, vreg_t *lhs,
                  vreg_t *arg1, vreg_t *arg2, vreg_t *arg3);
\end{verbatim}

The number \cc{op} not only indentifies the operation but also
contains the width it operates on, if applicable.

\begin{verbatim}
void emit_ir_extract (bblock_t *bblock, vreg_t *lhs, vreg_t *composite,
                      int start, int length);
void emit_ir_insert (bblock_t *bblock, vreg_t *lhs, vreg_t *template,
                     vreg_t *bits, int start, int length);
\end{verbatim}

Append an extract/insert operation to the end of \cc{bblock}.  May be
implemented as a macro.

\begin{verbatim}
void emit_ir_branch (bblock_t *bblock, bblock_t *target);
void emit_ir_conditional_branch (bblock_t *bblock, vreg_t *condition,
                                 bblock_t *target,
                                 bblock_t *alternative);
\end{verbatim}

Append an unconditional/conditional branch instruction to the end of
\cc{bblock}.  A branch ends a basic block, i.e., emitting another IR
instruction to the same basic block is illegal.

\begin{verbatim}
void emit_user_op (bblock_t *bblock, user_op_t *user_op,
                   vreg_t **args);
\end{verbatim}

Append a user-defined operation \cc{user\_op} to the end of
\cc{bblock}.  The arguments must be given as an array in \cc{args}.
An exit user-op ends a basic block, i.e., emitting another IR
instruction to the same basic block is illegal.

\subsection{Register Definition}
\label{sec:regdef}

\begin{verbatim}
reg_type_t make_register_type (char prefix);
\end{verbatim}

Make a new register type known.  \cc{prefix} is an arbitrary character
used for signifying the register type in debugging output.  It has no
other purpose.  This function must not be called after any host or
guest registers have been set up (see below), i.e., host and guest
registers must be set up after all register types have been
introduced.

\subsubsection{Host Registers}

\begin{verbatim}
void setup_hregs (reg_type_t type, unsigned int number);
\end{verbatim}

Specifies that the number of host registers of type \cc{type} is
\cc{number}.  This function must not be called for any type more than
once.  If the number of host registers for some type is not specified,
it is assumed to be zero.  Host registers are numbered separately for
each type, beginning with zero.

\begin{verbatim}
void set_hreg_name (reg_type_t type, unsigned int index, char *name);
\end{verbatim}

Sets the name of the host register of type \cc{type} with index
\cc{index} to \cc{name}.  The name is only used for debugging output.

\begin{verbatim}
hreg_t* get_hreg (reg_type_t type, unsigned int index);
\end{verbatim}

Returns the host register of type \cc{type} with index \cc{index}.

\subsubsection{Guest Registers}

\begin{verbatim}
void setup_gregs (reg_type_t type, unsigned int number);
\end{verbatim}

Specifies that the number of guest registers of type \cc{type} is
\cc{number}.  This function must not be called for any type more than
once\endnote{This is a restriction which will most definitely fall in
the future.  If we want to support compilation of higher-level
languages, guest registers correspond to local variables, and it is
obviously necessary to be able to compile fragments with different
numbers of local variables.  For this reason, we should also consider
making guest registers local to fragments, so that it's possible to
compile fragments with different guest registers at the same time.}.
If the number of host registers for some type is not specified, it is
assumed to be zero.  Host registers are numbered separately for each
type, beginning with zero.

\begin{verbatim}
void set_greg_location (reg_type_t type, unsigned int index,
                        hreg_t *base, off_t offset);
\end{verbatim}

Sets the home location of the guest register of type \cc{type} with
index \cc{index} to be at offset \cc{offset} from the host register
\cc{base}\endnote{Is this necessary when a fixed register mapping is
used?}.

\begin{verbatim}
void set_greg_name (reg_type_t type, unsigned int index, char *name);
\end{verbatim}

Sets the name of the guest register of type \cc{type} with index
\cc{index} to \cc{name}.  The name is only used for debugging output.

\begin{verbatim}
vreg_t* get_greg (reg_type_t type, unsigned int index);
\end{verbatim}

Returns the guest register of type \cc{type} with index \cc{index}.

\subsection{User-Defined Operations}

\begin{verbatim}
user_op_t* make_user_op (int id, unsigned int num_args, int is_exit,
                         int has_lhs, char *name);
\end{verbatim}

Introduces a new user-defined operation.  \cc{num\_args} specifies the
number of virtual register arguments the operation receives.  This
number must be at most \cc{4} and can be \cc{0}.  If the operation
produces a value, \cc{has\_lhs} must be non-zero, zero otherwise.  In
the former case, the virtual register the value is written to is
treated as a normal argument (and hence must be included in
\cc{num\_args}), which must be the first one.  The user-op's operation
must not depend on the value residing in the left-hand side virtual
register, i.e., it should only be written not, not read.  Similarly,
the user-op must not change the values of the other (right-hand side)
virtual registers.  \cc{is\_exit} must be non-zero iff the user-op is
an exit.  An exit always exits a fragment and must be the last IR
instruction in a basic block.  User-ops which call procedures which
return to the fragment are not exits.  The values of \cc{id} and
\cc{name} are of no consequence for code generation.  \cc{id} can be
used by the application to easily identify a user-op and \cc{name} is
used for debugging output.

\begin{verbatim}
void dispose_user_op (user_op_t *user_op);
\end{verbatim}

Frees the memory associated with the user-op \cc{user\_op}.  Fragments
containing \cc{user\_op} instructions are invalidated.

\begin{verbatim}
int user_op_id (user_op_t *user_op);
\end{verbatim}

Returns the \cc{id} of \cc{user\_op}.  May be implemented as a macro.

\subsection{Inspection}

\begin{verbatim}
bblock_t* fragment_entry (fragment_t *fragment);
\end{verbatim}

Returns the first basic block in \cc{fragment}, i.e., the (only) entry
point.

\begin{verbatim}
ir_insn_t* bblock_first_insn (bblock_t *bblock);
ir_insn_t* bblock_last_insn (bblock_t *bblock);
\end{verbatim}

Returns the first/last IR instruction in \cc{bblock}\endnote{The last
instruction is provided so that the block's successors can be
determined easily.}.

\begin{verbatim}
ir_insn_t* ir_insn_next (ir_insn_t *insn);
\end{verbatim}

Returns the successor instruction to \cc{insn} within its basic block.
Returns \cc{0} if \cc{insn} is the last instruction within its basic
block.

\begin{verbatim}
ir_op_t ir_insn_operator (ir_insn_t *insn);
int ir_operator_kind (ir_op_t op); /* macro */
int ir_operator_width (ir_op_t op); /* macro */
\end{verbatim}

\cc{ir\_insn\_operator} returns a value containing both the kind of the
operator (assign, integer add, shift left, $\ldots$) as well as the
width, if applicable.  The macros \cc{ir\_operator\_kind} and
\cc{ir\_operator\_width} extract these two out of the compound value.

\begin{verbatim}
vreg_t* ir_insn_vreg (ir_insn_t *insn, int n);
int ir_insn_field_start (ir_insn_t *insn);
int ir_insn_field_length (ir_insn_t *insn);
ir_int1_t ir_insn_int1_const (ir_insn_t *insn);
ir_int2_t ir_insn_int2_const (ir_insn_t *insn);
ir_int4_t ir_insn_int4_const (ir_insn_t *insn);
ir_int8_t ir_insn_int8_const (ir_insn_t *insn);
ir_float4_t ir_insn_float4_const (ir_insn_t *insn);
ir_float8_t ir_insn_float8_const (ir_insn_t *insn);
bblock_t* ir_insn_target (ir_insn_t *insn);
bblock_t* ir_insn_alternative_target (ir_insn_t *insn);
\end{verbatim}

Return operands of the IR instruction \cc{insn}.  \cc{ir\_insn\_vreg}
returns the \cc{n}th virtual register operand.  The count starts at
zero for the left hand side.  \cc{ir\_insn\_start} and
\cc{ir\_insn\_length} return the bit field start and length for
insert and extract instructions.  The \cc{ir\_insn\_XXX\_const}
functions return the constants for load constant instructions.
\cc{ir\_insn\_target} and \cc{ir\_insn\_alternative\_target} return
the target basic blocks for branches.  The latter is only applicable
on conditional branch instructions.  All of these may be implemented
as macros.

\subsection{Compilation}

\begin{verbatim}
void instruction_selection (fragment_t *fragment);
\end{verbatim}

Perform instruction selection on \cc{fragment}.  After this has been
done, changing the fragment's IR code is illegal (by emitting IR
instructions or adding basic blocks, for example).

\begin{verbatim}
size_t register_allocation (fragment_t *fragment,
                            hreg_t *spill_area_reg,
                            offset_t spill_area_offset,
                            int direction);
\end{verbatim}

Perform register allocation on \cc{fragment}.  Instruction selection
must already have been done on \cc{fragment}.  \cc{spill\_area\_reg}
is the host register used for addressing the spill area for temp
registers.  \cc{spill\_area\_offset} is the offset from that host
register to the start of the spill area, and \cc{direction} specifies
the direction in which the spill area extends (\cc{1} for increasing
offsets, \cc{-1} for decreasing offsets).

Returns the number of bytes used in the spill area, starting at the
start offset and extending in the specified direction.

\begin{verbatim}
size_t emit_host_code (fragment_t *fragment,
                       void *code_area, size_t code_area_size);
\end{verbatim}

Emits native code for \cc{fragment}.  Instruction selection and
register allocation must already have been done on \cc{fragment}.
\cc{code\_area} points to where the native code will be emitted.
\cc{code\_area\_size} gives the number of bytes that may maximally
be emitted.

Returns the number of bytes actually emitted, or a number larger than
\cc{code\_area\_size}, which indicates that the area was not large
enough.  In the latter case the actual amount returned does not
indicate how much additional space is needed.

\begingroup
\setlength{\parindent}{0pt}\setlength{\parskip}{2ex}
\renewcommand{\enotesize}{\normalsize}
\theendnotes\endgroup

\end{document}
